% Beautiful assignment LaTeX template — polished layout for reports
% Compile: make pdf  (or pdflatex + bibtex + pdflatex x2)
\documentclass[11pt,a4paper]{article}

% --- Typography & layout ---------------------------------------------------
\usepackage[utf8]{inputenc}
\usepackage[T1]{fontenc}
\usepackage{mathpazo}          % nicer serif font (Palatino)
\usepackage{microtype}
\usepackage[a4paper,margin=1in]{geometry}
\setlength{\headheight}{15pt} % avoids fancyhdr warning

% --- Useful packages -----------------------------------------------------
\usepackage{amsmath,amssymb,mathtools}
\usepackage{siunitx}
\usepackage{graphicx}
\usepackage{caption}
\usepackage{subcaption}
\usepackage{booktabs}
\usepackage{float}
\usepackage{tikz}              % drawing placeholders / decorations
\usepackage{enumitem}         % compact lists
\usepackage{xcolor}
\definecolor{accent}{HTML}{2A9D8F}
\definecolor{heading}{HTML}{264653}

% --- Section heading style ------------------------------------------------
\usepackage{titlesec}
\titleformat{\section}{\Large\bfseries\color{heading}}{\thesection}{1em}{}
\titleformat{\subsection}{\bfseries\color{heading}}{\thesubsection}{0.5em}{}
\titlespacing*{\section}{0pt}{12pt}{6pt}

% --- Code listing style ---------------------------------------------------
\usepackage{listings}
\lstdefinestyle{py}{
  language=Python,
  basicstyle=\ttfamily\small,
  backgroundcolor=\color{gray!6},
  frame=single,
  framesep=4pt,
  rulecolor=\color{gray!40},
  keywordstyle=\color{blue!65!black},
  commentstyle=\color{gray!55!black}\itshape,
  stringstyle=\color{red!65!black},
  showstringspaces=false,
  numbers=left,
  numberstyle=\tiny\color{gray},
  breaklines=true,
  captionpos=b,
}
\lstset{style=py}

% --- Pretty abstract box --------------------------------------------------
\usepackage{tcolorbox}
\tcbset{colback=gray!7, colframe=accent, left=6pt, right=6pt, boxrule=0.8pt}

% --- Header / footer -----------------------------------------------------
\usepackage{fancyhdr}
\pagestyle{fancy}
\fancyhf{}
\renewcommand{\headrulewidth}{0.6pt}
\renewcommand{\footrulewidth}{0.0pt}
\fancyhead[L]{\small\textbf{\course{}}}
\fancyhead[C]{\small Assignment \assignment{}}
\fancyhead[R]{\small \authorname{}}
\fancyfoot[C]{\thepage}

% --- Metadata (edit these) ------------------------------------------------
\newcommand{\course}{Trusted Artificial Intelligence}
\newcommand{\instructor}{Dr. Mostafa Tavasolipour}
\newcommand{\semester}{Spring 2024}
\newcommand{\assignment}{4}
\newcommand{\authorname}{taha majlesi - 810101504}
\newcommand{\studentid}{810101504}
\newcommand{\affiliation}{Department of Electrical and Computer Engineering, University of Tehran}

% --- Helpers --------------------------------------------------------------
\newcommand{\figref}[1]{Figure~\ref{#1}}
\newcommand{\tabref}[1]{Table~\ref{#1}}
\newcommand{\secref}[1]{Section~\ref{#1}}
\newcommand{\codefile}[1]{\lstinputlisting[style=py]{#1}}
\usepackage[hidelinks]{hyperref}
\usepackage{natbib}
\bibliographystyle{plainnat}

% --- Document -------------------------------------------------------------
\begin{document}

% ---------- Title page ----------------------------------------------------
\begin{titlepage}
  \centering
  {\LARGE\bfseries \course{} \par}
  \vspace{1.2cm}
  {\Huge\bfseries Homework \assignment{}\par}
  \vspace{0.6cm}
  {\large \semester{}\par}
  \vspace{1.2cm}
  {\Large\bfseries \authorname{}\par}
  \vspace{0.2cm}
  {\small ID: \studentid{} \quad | \quad \affiliation{}\par}
  \vspace{1.5cm}
  \begin{tikzpicture}
    \draw[accent,line width=2pt] (0,0) -- (8,0);
  \end{tikzpicture}
  \vfill
  {\large Instructor: \instructor{}\par}
  {\small Submitted: \today\par}
\end{titlepage}

% ---------- Abstract ------------------------------------------------------
\begin{tcolorbox}
\textbf{Abstract.} A short summary (max 200 words) describing the goal, main methods, and key results. Replace this with a concise overview of your experiments and findings.
\end{tcolorbox}

\vspace{6pt}
\tableofcontents
\clearpage

% ---------- Main sections -------------------------------------------------
\section{Introduction}
Brief problem statement, datasets, and objectives.

\section{Methods}
Describe model architecture, preprocessing, augmentations, and training protocol.
\subsection{Model}
Explain the architecture (e.g., custom ResNet18).
\subsection{Training setup}
Optimizers, lr schedule, batch size, epochs, and any regularization.

\section{Experiments}
Dataset splits, metrics, and evaluation protocol.
\subsection{Evaluation metrics}
Accuracy, loss, robustness, UMAP visualizations, etc.

\section{Results}
Quantitative tables and qualitative figures.

% include figures only if present, otherwise draw a placeholder
\begin{figure}[H]
  \centering
  \begin{subfigure}[b]{0.48\textwidth}
    \IfFileExists{figures/accuracy_plot.png}{\includegraphics[width=\textwidth]{figures/accuracy_plot.png}}{%
      \fbox{\parbox[c][4.3cm][c]{\textwidth}{\centering\small\itshape \texttt{accuracy\_plot.png} (missing)}} }
    \caption{Train / validation accuracy}
    \label{fig:acc}
  \end{subfigure}\hfill
  \begin{subfigure}[b]{0.48\textwidth}
    \IfFileExists{figures/loss_plot.png}{\includegraphics[width=\textwidth]{figures/loss_plot.png}}{%
      \fbox{\parbox[c][4.3cm][c]{\textwidth}{\centering\small\itshape \texttt{loss\_plot.png} (missing)}} }
    \caption{Train / validation loss}
    \label{fig:loss}
  \end{subfigure}
  \caption{Training curves (export your notebook plots to `template/figures/`).}
\end{figure}

% --- Summary results (placeholders filled from notebook demos) -----------------
\subsection{Summary of main results (placeholders)}
\begin{table}[H]
  \centering
  \caption{Fairness: Base vs Fair model (placeholders)}
  \begin{tabular}{lccc}
    \toprule
    Model & Accuracy & Disparate Impact & Zemel-proxy \\
    \midrule
    Base classifier (Logistic) & 0.85 (placeholder) & 0.70 (placeholder) & 0.12 (placeholder) \\
    Fair model (Promotion/Demotion) & 0.83 (placeholder) & 0.95 (placeholder) & 0.06 (placeholder) \\
    \bottomrule
  \end{tabular}
\end{table}

\subsection{Neural Cleanse (Q1) — reconstruction summary}
\begin{figure}[H]
  \centering
  \IfFileExists{figures/trigger_reconstructed.png}{\includegraphics[width=0.45\textwidth]{figures/trigger_reconstructed.png}}{%
    \fbox{\parbox[c][4.3cm][c]{0.45\textwidth}{\centering\small\itshape \texttt{trigger\_reconstructed.png} (missing)}} }
  \caption{Reconstructed trigger for detected attacked class (placeholder image).}
  \label{fig:trigger}
\end{figure}

\begin{table}[H]
  \centering
  \caption{Neural Cleanse: model metrics (placeholders)}
  \begin{tabular}{lcc}
    \toprule
    Metric & Before cleansing & After unlearning \\
    \midrule
    Clean accuracy & 0.99 (placeholder) & 0.98 (placeholder) \\
    Attack Success Rate (ASR) & 0.92 (placeholder) & 0.05 (placeholder) \\
    \bottomrule
  \end{tabular}
\end{table}

\subsection{Privacy (Q2) — Laplace examples}
Example Laplace-calculation results are shown in the notebook (see `code/notebook.ipynb`). Placeholder values used in the report: average-income (noisy) = 50234.1 (placeholder).

% --- end of results placeholders -------------------------------------------------

\section{Discussion}
Interpret the results and propose next steps.

\section{Conclusion}
Concise takeaways.

% ---------- Appendix ------------------------------------------------------
\appendix
\section{Hyperparameters and Implementation Details}
Full hyperparameters and experimental settings.

\section{Selected Code}
Important snippets or reference to script files.
\begin{lstlisting}[style=py,caption={Training loop (example)}]
for epoch in range(epochs):
    model.train()
    for x,y in train_loader:
        # training step
        pass
\end{lstlisting}

\section{Additional Figures}
Extra visualizations, UMAP plots, adversarial examples, etc.

\clearpage
\bibliography{references}
\end{document}
