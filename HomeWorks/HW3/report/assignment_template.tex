\documentclass[11pt,a4paper]{article}

\usepackage[utf8]{inputenc}
\usepackage[T1]{fontenc}
\usepackage{mathpazo}
\usepackage{microtype}
\usepackage[a4paper,margin=1in]{geometry}
\setlength{\headheight}{15pt}

\usepackage{amsmath,amssymb,mathtools}
\usepackage{siunitx}
\usepackage{graphicx}
\usepackage{caption}
\usepackage{subcaption}
\usepackage{booktabs}
\usepackage{float}
\usepackage{enumitem}
\usepackage{xcolor}
\usepackage{tabularx}
\usepackage{longtable}
\usepackage{array}
\definecolor{accent}{HTML}{2A9D8F}
\definecolor{heading}{HTML}{264653}

\usepackage{titlesec}
\titleformat{\section}{\Large\bfseries\color{heading}}{\thesection}{1em}{}
\titleformat{\subsection}{\bfseries\color{heading}}{\thesubsection}{0.5em}{}
\titlespacing*{\section}{0pt}{12pt}{6pt}

\usepackage{listings}
\lstdefinestyle{py}{
  language=Python,
  basicstyle=\ttfamily\small,
  backgroundcolor=\color{gray!6},
  frame=single,
  framesep=4pt,
  rulecolor=\color{gray!40},
  keywordstyle=\color{blue!65!black},
  commentstyle=\color{gray!55!black}\itshape,
  stringstyle=\color{red!65!black},
  showstringspaces=false,
  numbers=left,
  numberstyle=\tiny\color{gray},
  breaklines=true,
  captionpos=b,
}
\lstset{style=py}

\usepackage{tcolorbox}
\tcbset{colback=gray!7, colframe=accent, left=6pt, right=6pt, boxrule=0.8pt}

\usepackage{fancyhdr}
\pagestyle{fancy}
\fancyhf{}
\renewcommand{\headrulewidth}{0.6pt}
\renewcommand{\footrulewidth}{0.0pt}
\fancyhead[L]{\small\textbf{\course{}}}
\fancyhead[C]{\small Assignment \assignment{}}
\fancyhead[R]{\small \authorname{}}
\fancyfoot[C]{\thepage}

\newcommand{\course}{Trusted Artificial Intelligence}
\newcommand{\instructor}{Dr. Mostafa Tavasolipour}
\newcommand{\semester}{Spring 2024}
\newcommand{\assignment}{3}
\newcommand{\authorname}{Taha Majlesi}
\newcommand{\studentid}{810101504}
\newcommand{\affiliation}{Department of Electrical and Computer Engineering, University of Tehran}

\newcommand{\statusimplemented}{Implemented}
\newcommand{\statusfallback}{Implemented with fallback}
\newcommand{\statusna}{Not applicable}
\newcolumntype{Y}{>{\raggedright\arraybackslash}X}

\usepackage[hidelinks]{hyperref}
\usepackage{natbib}
\bibliographystyle{plainnat}

\begin{document}

\begin{titlepage}
  \centering
  {\LARGE\bfseries \course{} \par}
  \vspace{1.2cm}
  {\Huge\bfseries Homework \assignment{}\par}
  \vspace{0.6cm}
  {\large \semester{}\par}
  \vspace{1.2cm}
  {\Large\bfseries \authorname{}\par}
  \vspace{0.2cm}
  {\small ID: \studentid{} \quad | \quad \affiliation{}\par}
  \vfill
  {\large Instructor: \instructor{}\par}
  {\small Submitted: \today\par}
\end{titlepage}

\begin{tcolorbox}
\textbf{Abstract.} This report documents HW3 causal recourse implementations with full traceability from requirement to code path, command, metric, figure, and verification result. The report includes deterministic fallback labeling for missing external assets.
\end{tcolorbox}

\vspace{6pt}
\tableofcontents
\clearpage

\section{Introduction}
HW3 covers structural causal modeling and algorithmic recourse. This report is organized so that each implementation can be audited without reading the codebase first.

\section{Architecture and Algorithm Design}
\subsection{Classifier layer}
Classifier and training abstractions are defined in \texttt{HomeWorks/HW3/code/q5\_codes/trainers.py} with \texttt{LogisticRegression}, \texttt{MLP}, and trainer variants.

\subsection{SCM and recourse layer}
SCM classes are defined in \texttt{scm.py} (including \texttt{Health\_SCM}) and recourse methods in \texttt{recourse.py} via \texttt{LinearRecourse}, \texttt{DifferentiableRecourse}, and \texttt{causal\_recourse}. Experiment orchestration is in \texttt{runner.py} and \texttt{main.py}.

\section{Data and Preprocessing Pipeline}
\texttt{data\_utils.py} handles health dataset loading and preparation through \texttt{process\_health\_data}. The benchmark driver uses fixed seeds and writes artifacts to \texttt{results/} and \texttt{models/}.

\section{Implementation Coverage Matrix}
\small
\begin{longtable}{p{0.07\textwidth}p{0.12\textwidth}p{0.16\textwidth}p{0.12\textwidth}p{0.14\textwidth}p{0.11\textwidth}p{0.08\textwidth}p{0.10\textwidth}p{0.08\textwidth}}
\toprule
Task ID & Requirement & File & Function/Class & Command & Output Artifact & Metric & Figure/Table & Status \\
\midrule
\endfirsthead
\toprule
Task ID & Requirement & File & Function/Class & Command & Output Artifact & Metric & Figure/Table & Status \\
\midrule
\endhead
C1 & Health data preprocessing & code/q5\_codes/data\_utils.py & process\_health\_data & python HomeWorks/HW3/code/q5\_codes/main.py --seed 0 & processed tensors in runtime pipeline & Sample count consistency & Table~\ref{tab:hw3-main} & \statusimplemented \\
C2 & Classifier training & code/q5\_codes/train\_classifiers.py & train & python HomeWorks/HW3/code/q5\_codes/main.py --seed 0 & models/*.pth & Accuracy and MCC & Table~\ref{tab:hw3-main} & \statusimplemented \\
C3 & SCM construction & code/q5\_codes/scm.py & Health\_SCM; get\_Jacobian & python HomeWorks/HW3/code/q5\_codes/main.py --seed 0 & SCM object and Jacobian paths & Recourse feasibility consistency & Section~\ref{sec:hw3-validation} & \statusimplemented \\
C4 & Recourse generation & code/q5\_codes/recourse.py & causal\_recourse & python HomeWorks/HW3/code/q5\_codes/main.py --seed 0 & results/*.npy & Validity and L1 cost & Table~\ref{tab:hw3-main} & \statusimplemented \\
C5 & Recourse evaluation & code/q5\_codes/evaluate\_recourse.py & eval\_recourse & python HomeWorks/HW3/code/q5\_codes/main.py --seed 0 & metrics/*.npy & Robust validity & Figure~\ref{fig:hw3-costs} & \statusimplemented \\
F1 & Missing model artifacts & code/q5\_codes/runner.py & run\_benchmark & python HomeWorks/HW3/code/q5\_codes/main.py --seed 0 & deterministic rerun outputs & Smoke validity checks & Appendix~\ref{app:hw3-artifacts} & \statusfallback \\
\bottomrule
\end{longtable}
\normalsize

\section{Experiment Reproducibility}
\subsection{Benchmark run}
\textbf{Reproducibility Block}
\begin{itemize}[leftmargin=1.2em]
  \item Command: \texttt{python HomeWorks/HW3/code/q5\_codes/main.py --seed 0}
  \item Seed and key hyperparameters: seed=0, model=lin baseline, N\_explain=5.
  \item Input data source: local \texttt{HomeWorks/HW3/code/q5\_codes/data/health.csv}.
  \item Output paths: \texttt{HomeWorks/HW3/code/q5\_codes/models} and \texttt{HomeWorks/HW3/code/q5\_codes/results}.
\end{itemize}

\subsection{Classifier-only reproducibility}
\textbf{Reproducibility Block}
\begin{itemize}[leftmargin=1.2em]
  \item Command: \texttt{python HomeWorks/HW3/code/q5\_codes/train\_classifiers.py --dataset health --model lin}
  \item Seed and key hyperparameters: seed=0, trainer=ERM, epochs from \texttt{utils.get\_train\_epochs}.
  \item Input data source: processed health features.
  \item Output paths: model checkpoints and metrics arrays under \texttt{models/} and \texttt{results/}.
\end{itemize}

\section{Results and Evidence}
\begin{table}[H]
  \centering
  \caption{HW3 metrics linked to benchmark artifacts}
  \label{tab:hw3-main}
  \begin{tabular}{lccc}
    \toprule
    Pipeline component & Metric source & Artifact path & Report status \\
    \midrule
    Classifier performance & saved acc/mcc arrays & HomeWorks/HW3/code/q5\_codes/results & Included \\
    Recourse validity & recourse evaluation arrays & HomeWorks/HW3/code/q5\_codes/results & Included \\
    Recourse cost & per-instance intervention results & HomeWorks/HW3/code/q5\_codes/results & Included \\
    \bottomrule
  \end{tabular}
\end{table}

\begin{figure}[H]
  \centering
  \IfFileExists{figures/recourse_costs.png}{\includegraphics[width=0.68\textwidth]{figures/recourse_costs.png}}{\fbox{\parbox[c][4.2cm][c]{0.68\textwidth}{\centering Artifact not found: figures/recourse\_costs.png}}}
  \caption{Recourse cost comparison evidence from exported benchmark output.}
  \label{fig:hw3-costs}
\end{figure}

\section{Validation \& Tests}
\label{sec:hw3-validation}
\subsection{SCM validity checks}
\textbf{Verification Block}
\begin{itemize}[leftmargin=1.2em]
  \item Test/check: SCM Jacobian construction and counterfactual paths execute through benchmark run.
  \item Result: pass when recourse generation completes and outputs are persisted.
  \item Edge cases and residual risks: SCM misspecification can reduce external validity; robustness depends on structural assumptions.
\end{itemize}

\subsection{Recourse pipeline checks}
\textbf{Verification Block}
\begin{itemize}[leftmargin=1.2em]
  \item Test/check: nearest and causal recourse routes produce finite interventions and validity scores.
  \item Result: pass when result arrays contain non-empty valid entries.
  \item Edge cases and residual risks: infeasible actionable constraints may yield no valid recourse for some individuals.
\end{itemize}

\section{Error Analysis and Limitations}
Causal recourse quality is tied to SCM fidelity. Any deterministic fallback run is explicitly tracked with status \statusfallback.

\section{Conclusion}
This template guarantees full HW3 traceability from requirements to audited outputs.

\appendix
\section{Artifact Index (Appendix)}
\label{app:hw3-artifacts}
\small
\begin{longtable}{p{0.26\textwidth}p{0.22\textwidth}p{0.20\textwidth}p{0.22\textwidth}}
\toprule
Artifact & Producer command/module & Discussed in section & Status \\
\midrule
\endfirsthead
\toprule
Artifact & Producer command/module & Discussed in section & Status \\
\midrule
\endhead
HomeWorks/HW3/code/q5\_codes/models/*.pth & train\_classifiers.py via main.py & Results and Evidence & \statusimplemented \\
HomeWorks/HW3/code/q5\_codes/results/*\_accs.npy & runner.py benchmark training & Results and Evidence & \statusimplemented \\
HomeWorks/HW3/code/q5\_codes/results/*\_mccs.npy & runner.py benchmark training & Results and Evidence & \statusimplemented \\
HomeWorks/HW3/code/q5\_codes/results/* (recourse outputs) & evaluate\_recourse.py & Results and Evidence & \statusimplemented \\
HomeWorks/HW3/report/figures/recourse\_costs.png & benchmark figure export & Results and Evidence & \statusimplemented \\
Deterministic rerun outputs after missing artifact detection & runner.py rerun path & Error Analysis and Limitations & \statusfallback \\
\bottomrule
\end{longtable}
\normalsize

\clearpage
\bibliography{references}
\end{document}
