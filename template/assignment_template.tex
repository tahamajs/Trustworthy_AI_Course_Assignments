% Assignment LaTeX template — use for all homeworks
% Compile: make pdf  (or pdflatex + bibtex + pdflatex x2)
\documentclass[12pt,a4paper]{article}

% Page layout
\usepackage[utf8]{inputenc}
\usepackage[T1]{fontenc}
\usepackage{lmodern}
\usepackage{microtype}
\usepackage[a4paper,margin=1in]{geometry}
\usepackage{setspace}
\onehalfspacing

% Math & symbols
\usepackage{amsmath,amssymb,mathtools}
\usepackage{siunitx}

% Figures, tables, captions
\usepackage{graphicx}
\usepackage{caption}
\usepackage{subcaption}
\usepackage{booktabs}
\usepackage{multirow}
\usepackage{float}

% Code listings
\usepackage{listings}
\usepackage{xcolor}
\lstset{
  basicstyle=\ttfamily\small,
  breaklines=true,
  frame=single,
  backgroundcolor=\color{gray!7},
  keywordstyle=\color{blue},
  commentstyle=\color{green!50!black},
  stringstyle=\color{red!60!black},
  showstringspaces=false,
  numbers=left,
  numberstyle=\tiny\color{gray},
  stepnumber=1,
  numbersep=6pt,
}

% Bibliography
\usepackage{natbib}
\bibliographystyle{plainnat}

% Hyperlinks
\usepackage[hidelinks]{hyperref}

% Header / footer
\usepackage{fancyhdr}
\pagestyle{fancy}
\fancyhf{}
\lhead{\course{} -- Assignment \assignment{}}
\rhead{\authorname{}} 
\cfoot{\thepage}

% Short metadata commands (customise in document preamble)
\newcommand{\course}{Trusted Artificial Intelligence}
\newcommand{\instructor}{Dr. Mostafa Tavasolipour}
\newcommand{\semester}{Spring 2024}
\newcommand{\assignment}{1}
\newcommand{\authorname}{Your Name}
\newcommand{\studentid}{StudentID}
\newcommand{\affiliation}{Department of Electrical and Computer Engineering, University of Tehran}

% Helpful macros
\newcommand{\figref}[1]{Figure~\ref{#1}}
\newcommand{\tabref}[1]{Table~\ref{#1}}
\newcommand{\secref}[1]{Section~\ref{#1}}
\newcommand{\codefile}[1]{\lstinputlisting[language=Python]{#1}}

% Document starts
\begin{document}

% Title page
\begin{titlepage}
  \centering
  {\scshape\LARGE University of Tehran \par}
  \vspace{1.5cm}
  {\huge\bfseries Homework \assignment{}: Project Report\par}
  \vspace{1.0cm}
  {\Large \course{}\par}
  \vspace{1.0cm}
  {\large Instructor: \instructor{}\par}
  \vspace{1.0cm}
  {\Large \authorname{} \quad | \quad ID: \studentid{}\par}
  \vfill
  {\small Submitted: \today\par}
\end{titlepage}

\begin{abstract}
A concise abstract (max 200 words) summarising objectives, methods, and main results. Replace this with a short summary of your experiments and findings.
\end{abstract}

\tableofcontents
\clearpage

\section{Introduction}
State the problem, datasets used, and the goal of the assignment. Keep this brief and focused.

\section{Methods}
Describe model architectures, training procedure, hyperparameters, data preprocessing and augmentations. Use subsections for clarity.
\subsection{Model}
Describe network architecture (e.g., ResNet18 — custom implementation).
\subsection{Training setup}
Optimizer, learning rate, batch size, number of epochs, and any regularizers.

\section{Experiments}
Explain datasets and experimental protocol (train/test splits, metrics).
\subsection{Evaluation metrics}
Accuracy, loss, robustness metrics, or others you used.

\section{Results}
Present quantitative and qualitative results. Include plots exported from your notebooks.

\begin{figure}[H]
  \centering
  \begin{subfigure}[b]{0.48\textwidth}
    \includegraphics[width=\textwidth]{figures/accuracy_plot.png}
    \caption{Train / validation accuracy}
    \label{fig:acc}
  \end{subfigure}
  \begin{subfigure}[b]{0.48\textwidth}
    \includegraphics[width=\textwidth]{figures/loss_plot.png}
    \caption{Train / validation loss}
    \label{fig:loss}
  \end{subfigure}
  \caption{Example training curves — export plots from your notebook into the `figures/` folder.}
\end{figure}

\begin{table}[H]
  \centering
  \caption{Summary of main results}
  \begin{tabular}{lcc}
    \toprule
    Model & Test accuracy (SVHN) & Test accuracy (MNIST) \\
    \midrule
    Baseline ResNet18 & 0.00 & 0.00 \\
    +Augmentations & 0.00 & 0.00 \\
    Pretrained Feat. Extractor & 0.00 & 0.00 \\
    \bottomrule
  \end{tabular}
\end{table}

\section{Discussion}
Interpret the results, strengths/weaknesses, and propose next steps.

\section{Conclusion}
One-paragraph takeaway summary of what you did and the main findings.

\section*{Acknowledgements}
(Optional) Acknowledge help, datasets or libraries used.

\clearpage
\appendix
\section{Hyperparameters and Implementation Details}
Provide tables or bullet lists with full hyperparameter settings.

\section{Selected Code}
Include important snippets or point to code files. Example: full training loop excerpt.
\begin{lstlisting}[language=Python, caption={Training loop (example)}]
for epoch in range(epochs):
    model.train()
    for x,y in train_loader:
        # training step
        pass
\end{lstlisting}

% To include a full file (e.g., train.py) use \codefile{path/to/train.py}

\section{Additional Figures}
Any extra plots, UMAP visualizations, or adversarial examples can be included here.

\clearpage
\bibliography{references}

\end{document}
